\begin{solution} 
Supongamos $p=q$

\begin{align}
|Z(G)| \equiv |G| \equiv p^2 \equiv 0 (mod p) &\implies p \mid  |Z(G)|\\
|G/Z(G)| = 
  \begin{cases}
   1 & \text{si } |Z(G)|=p^2 \\
   p & \text{si } |Z(G)|=p
  \end{cases} &\implies \text{$G/Z(G)$ es abeliano} \implies \text{$G/Z(G)$ es resoluble}\\
 \left.\begin{aligned}
		Z(G) \triangleleft G\\
		\text{$Z(G)$ resoluble}\\
		\text{$G/Z(G)$ resoluble}\\
       \end{aligned}
 \right\}
 &\implies \text{$G$ es resoluble}
\end{align}

Supongamos ahora que $p \neq q$. Sin perdida de generalidad, $p < q$. Sean $S_q$ un q-subgrupo de Sylow y $n_q$ la cantidad de estos:

\begin{align}
 \left.\begin{aligned}
		n_q &\equiv 1 (mod q)\\
		n_q &\mid [G:S_q]=p
       \end{aligned}
 \right\}
 &\implies n_q = 1 \iff S_q \triangleleft G\\
|S_q| = q \implies \text{$S_q$ es ciclico} &\implies \text{$S_q$ es abeliano} \implies \text{$S_q$ es resoluble}\\
|G/S_q| = p \implies \text{$G/S_q$ es ciclico} &\implies \text{$G/S_q$ es abeliano} \implies \text{$G/S_q$ es resoluble}\\
 \left.\begin{aligned}
		&S_q \triangleleft G\\
		&\text{$S_q$ resoluble}\\
		&\text{$G/S_q$ resoluble}\\
       \end{aligned}
 \right\}
 &\implies \text{$G$ es resoluble}\\
 &\Box
\end{align}

\end{solution}