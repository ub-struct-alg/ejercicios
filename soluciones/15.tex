\begin{solution}
\\Toda permutación descompone en producto de transposiciones. Esta descomposición no es única, más su $\varepsilon$ se mantiene. Es decir, son o pares o impares.\\
Sea la aplicación: \\
\begin{align}
f: P \to I \\
\sigma \mapsto \tau\sigma
\end{align}
Que envia las permutaciones pares a las impares, solo queda ver que esta es biyectiva.\\
\begin{enumerate}
\item inyectiva: \\
\begin{align}
f(\sigma) = f(\xi) \implies \sigma = \xi ?\\
\tau\sigma = \tau\xi \implies \sigma= \xi 
\end{align}
\item exhaustiva: \\
    Si, pues $f(\tau\sigma)= \tau\tau\sigma=\sigma$.
\end{enumerate}
Como $f$ es biyectiva, entonces tendrá siempre la misma cantidad de permutaciones pares que impares; ergo la mitad de $S_n$.
\end{solution}
