\begin{solution} 
Factorizamos $15=3\cdot5$ y aplicamos el 3r y 2º Teoremas de Sylow:

\begin{equation}
 \left.\begin{aligned}
		|S_3| &= 3\\
		n_3 &\equiv 1 (mod 3)\\
		n_3 &\mid [G:S_3]=5
       \end{aligned}
 \right\}
 \implies n_3 = 1 \iff S_3 \triangleleft G
\end{equation}
\begin{equation} 
 \left.\begin{aligned}
		|S_5| &= 5\\
		n_5 &\equiv 1 (mod 5)\\
		n_5 &\mid [G:S_5]=3
       \end{aligned}
 \right\}
 \implies n_5 = 1 \iff S_5 \triangleleft G
\end{equation}

Donde $S_p$ es un p-subgrupo de Sylow de $G$ y $n_p$ la cantidad de estos.\\
$S_3$ y $S_5$ son ciclicos por ser de orden primo. Tomemos $S_3=\langle a \rangle$ y $S_5=\langle b \rangle$.\\
Observemos que todos los elementos de $S_3$ y $S_5$, exceptuando el neutro, son generadores de estos. De lo que se deduce inmediatamente que $S_3 \cap S_5 = \{e\}$.\\
Veamos que $a$ y $b$ conmutan:

\begin{align} 
ab=ba &\iff aba^{-1}b^{-1} = e \\
\left.\begin{aligned}
	\underbrace{a b a^{-1}}_{\in S_5 \Leftarrow S_5 \triangleleft G} b^{-1} &\in S_5 \\
	a\underbrace{b a^{-1} b^{-1}}_{\in S_3 \Leftarrow S_3 \triangleleft G} &\in S_3
   \end{aligned}
\right\}
&\implies aba^{-1}b^{-1} \in S_3 \cap S_5 = \{e\} \\
ord(ab) = mcm(ord(a), ord(b)) &= mcm(3, 5) = 15\\
ord(ab) = |G| &\implies G = \langle ab \rangle\\
&\Box
\end{align}


\end{solution}