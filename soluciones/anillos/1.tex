\begin{solution}\\
Tenemos que $a \in A$ es unidad.
\begin{align}
\exists b \in A \mid ab = 0 
&\implies 0 = a^{-1}0 = a^{-1}(ab) = 1b = b \implies\\
b = 0 &\iff a \text{ no es un divisor de cero}\\
&\Box
\end{align}

Tenemos que $I \subset A$ es un ideal.
\begin{align}
\exists u \in I \mid u \in A^* &\implies u^{-1}u = 1 \in I\\
a \in A \implies a = a \cdot 1 \in I &\implies I=A\\
&\Box
\end{align}

Tenemos $a \in A$ y $u \in A^*$.
\begin{align}
u \in A \land a \in (a)
&\implies ua \in (a) 
\implies (ua) \subset (a)\\
u^{-1} \in A \land ua \in (ua)
&\implies u^{-1}ua = a \in (ua) 
\implies (a) \subset (ua)\\
(ua) \subset (a) \land (a) \subset (ua) &\iff (ua) = (a)\\
&\Box
\end{align}

Tenemos que $A$ es un dominio de integridad y $a,b \in A$. \\
Ya hemos demostrado en 1.c $\exists u \in A^* \mid b = au \implies (a) = (b)$.\\
Solo queda demostrar $(a) = (b) \implies \exists u \in A^* \mid b = au$:
\begin{align}
(a) = (b) \implies a \in (b) &\implies \exists c \in A \mid a = bc \\
(a) = (b) \implies b \in (a) &\implies \exists d \in A \mid b = ad \\
b = ad = bcd \implies b(1 - cd) = 0 
&\xRightarrow[\text{de integridad}]{\text{A es dominio}}
\begin{cases}
b = 0 \implies a = bc = 0 & u:=1\\
\lor\\
1 = cd \implies d \in A^* & u:=d
\end{cases}\\
&\Box
\end{align}

\end{solution}