\begin{solution}
\begin{align}
m&:= gr(f)\\
n&:= gr(g)\\
f^{(m)}&:= f\\
f^{(i)}_j &:= \text{coeficiente $j$ de $f^{(i)}$}\\
\end{align}
Empleamos el algoritmo de división tal cual:
\begin{align}
&i:=m\\
&\text{mientras $i \ge n$ hacer:}\\
&\quad f^{(i-1)}(x) := f^{(i)}(x) - f^{(i)}_ix^{i-n}g(x)\\
&\quad i:= i - 1\\
&q:=\sum_{i=n}^{m} f^{(i)}_ix^{i-n}\\
&r:=f^{(n-1)}
\end{align}
Encontramos $q,r$ por lo que demostramos su existencia. 
$q$ es unico porque el primer coeficiente de $q$ y $f$ tiene que se el mismo, 
por lo que el primer coeficiente de $q$ es unico. 
Aplicamos recursión y obtenemos que todos los coeficientes de $q$ son unicos,
por lo que tenemos que $q$ es unico. 
Inmediatamente, $r := f-qg$ tambien es unico. 
Queda demostrada la unicidad de $q$ y $r$.$\Box$

Como ejemplo de no existencia tenemos $3,2\in\mathbb{Z}$ ya que 
el residuo debe de ser 0 y eso solo ocurre con
el cociente perteneciente a $\mathbb{Q}\setminus\mathbb{Z}$.\\
Como ejemplo de existencia pero no unicidad tomemos en $\mathbb{Z}/8\mathbb{Z}$,
$f:=6x^2$ y $g:=2x$. Tenemos almenos 2 soluciones distintas:
\begin{align}
q&=3x\\
r&=0\\
2x3x+0 &= 6x^2\\
q'&=4x^3+3x\\
r'&=0\\
2x(4x^3+3x)+0 &= 8x^4 + 6x^2 = 0 + 6x = 6x^2 
\end{align}
\end{solution}