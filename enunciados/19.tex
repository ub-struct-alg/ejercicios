\question 
El grupo diedral $D_{2,n}$ es el grupo de los desplazamientos en el plano que dejan invariante un poligono regular de $n$ lados. Esto es, $D_{2,n}=\langle \rho, \sigma \rangle$, donde $\rho$ es una rotación de angulo $\frac{2\pi}{n}$ centrada en el centro de simetria del poligono y $\sigma$ es una simetria axial respecto a auno de los radios del poligono.
\begin{parts}
\part Demuestra ${\rho^n=\sigma^2=Id}\land{\rho\sigma=\sigma\rho^{-1}}$.
\part Escribe todos los elementos de $D_{2,n}$ ¿Cuantos son?
\part Define un monomorfismo $f: D_{2,n} \to S_n$.
\part Demuestra $\neg(D_{2,4} \simeq \mathbb{Z}/8\mathbb{Z})$.
\end{parts}