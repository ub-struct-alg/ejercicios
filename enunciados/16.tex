\question
\begin{parts}
\part Demuestra $\forall \sigma \in S_n \quad \sigma \circ (a_1,\dotsc,a_r)
\circ\sigma^{-1}=(\sigma(a_1),\dotsc,\sigma(a_r))$.
\part Demuestra $\forall \sigma_1, \sigma_2 \in S_n \quad (ord(\sigma_1)=
ord(\sigma_2)\implies \exists {\sigma \in S_n} | 
{\sigma \circ \sigma_1 \circ \sigma^{-1} = \sigma_2})$.
\part Sean $\sigma_1,\dotsc,\sigma_k \in S_n$ ciclos disjuntos dos a dos y 
también $\tau_1,\dotsc,\tau_k \in S_n$ ciclos disjuntos dos a dos. Pongamos 
$\sigma:=\sigma_1\circ\dotsb\circ\sigma_k$ y $\tau:=\tau_1\circ\dotsb\circ\tau_k$. 
Demuestra $\forall i $ $ 1\leq i \leq k$, si la longitud del ciclo $\sigma_i$ 
coincide con la del ciclo $\tau_i \implies \exists \rho \in S_n \mid 
\rho\circ\sigma\circ\rho^-1 = \tau$
\end{parts}
